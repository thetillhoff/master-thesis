\chapter{Outlook}

% aspects of mdsd workflow:
% - always:
%   - define modeling language (abstract and concrete syntax)
%   - tool support to define models
%   - persist models in files or dbs
%   - tranform models to text
% - often:
%   - transform model to model
%   - transform text to model
%   - analyze and interpret models
%   - automate workflows from single steps

% why meta models: allows to reuse technology for
% - storing models in files
% - storing models in dbs
% - creating a form-based editor
% - creating a graphical editor
% - generating code from models
% - transforming models into other models
% metamodel is used to describe abstract syntax, not concrete syntax!

\begin{itemize}
  \item compare metamodels, find similarities, could lead to common ground
  \item vendor bioses should support http by default or embed ipxe for network boot. Maybe even allow flashing the network-boot system (remotely?). VMware does support this already for its VMs: https://ipxe.org/howto/vmware
  \item ipmi like interface for provisioning, f.e. provide (remote) kernel path and parameters (addition to "local boot", "net boot" selection)
  \item common standard for bmc/ipmi features.
  \item making all bios settings available over an scriptable interface - bmc is not (universal) enough.
  \item tosca standard improvements from notes. Example: Two types of script execution: one on the orchestrator and one on the nodes.
\end{itemize}
