\chapter{Conclusion}

% - 2-4 pages
% - do NOT summarize the thesis -> thats what the abstract is for. (but you can summarize most important results as introduction)
% - synthesize findings and conclude what can be learnt from them
% - refer to research questions and answer them (confirm, rejected). can be sections
% - clarify whether result/software meets requirements as stated earlier.
% - compare with results/software from others

%%%%% start writing here

% initial problems
% - provider migrations are hard
% - orchestrators do the same thing, but have each their own implementation
% - bootstrapping is always offloaded and not part of the orchestrator

% repeat research questions, what are their answers?
% - can bare-metal machines be deployed on-demand like virtual workloads on providers?
% - Is it possible to do so without the requirement of an always-on operator (thus removing the initial bootstrapping effort)?
% - How can hardware constraints be mirrored in an iac language

At the beginning of this thesis, it was unclear, whether bare-metal machines can be deployed on-demand and without an always-on operator. With the hardware extension for \gls{toscaacr} and the introduced command-line tool, it is now clear that this is possible.
\newline
The extension also showed how hardware could be modeled in an \gls{iacacr} language and how it blends in with the existing definitions.
\newline
Additionally, the hardware detection step with the live-\gls{osacr} made it clear that requirements regarding the hardware (like the minimum amount of RAM required for a certain application) can be fulfilled at runtime, too.
\newline\smallskip
% what was the new approach of this thesis?
The command-line tool approach that the thesis took was initially meant to leave things simple, but as it turned out, it brought many benefits over the always-on approach other tools have. Not only can it just run on-demand, but it doesn't require any kind of (infrastructure-)bootstrapping before it can get to the actual work.
\newline
% compare with results from others (other provisiong software)
% ... Can combine provisioning with iac, which was formerly not possible
On one side, its on-demand nature and the low initial requirements set this tool greatly apart from other hardware-provisioning tools. On the other side, making the ability to describe and provision hardware part of the orchestrator with an admittedly relatively common \gls{iacacr} capability set it apart from other \gls{iacacr} tools. The result is something that did not exist previously and has the potential to initiate a new breed of tools that broaden the infrastructure part of \gls{iacacr}.
\newline\smallskip
The \gls{toscaacr} standard was found to be extremely flexible and easy to integrate with other tools - which are not necessarily part of the orchestrator. The two Golang packages around \gls{csaracr} and \gls{toscaacr} are also mentionable successes along the way and might help the standard to increase its userbase.
\newline
% own work: additional properties, culling
During the \gls{dslacr} finding phase, additional dimensions were developed that are important when comparing such languages. While the listed dimensions help select a language, this thesis made it also clear that they are (still) not the only means required to be looked at for choosing the right language for specific use cases.
\newline
While working with the \gls{toscaacr} specification, several smaller issues in the document itself were found. But the work also highlighted some larger issues about the design of the standard, that would drastically simplify it and decrease the initially required learning curve.
