\chapter{Analysis}

% - 5-10 pages
% - case studies - how does it work in close to real-world settings
% - how well does it work (scale, speed, stability)

% todo reference architecture?

% existing infrastructure
%   up-to-date tosca orchestrator, compliant with the original standard and Simple-Profile -> IaC tool at least
%   provisioning of new hardware is possible
%   migration to tosca necessary

% datacenter from scratch / initial deployment
%   describe both physical and virtual infrastructure with the same language
%   apart from that (and phyiscal movement / cabling), no manual work required -> no installing of operating systems, etc.
%   can be used as base for other iac tools like ansible.

% limitations

% scaling: disk read speed (live-os image), network speed of root node
% speed: deploy without installation; available within a minute. deploy with installation; same amount plus os installation.
% stability: recovery feature to be done. example solution: ipmi for remote power-off



Assuming a scenario where a new datacenter should be provisioned from scratch (no \gls{osacr} deployed), the tool described in this thesis can automate the initial setup. Because \gls{toscaacr} is the language of choice, and a compatible orchestrator is embedded in the application, users can make use of all capabilities the \gls{dslacr} provides. Since the original language can already be used for further provisioning and configuration management, the resulting application is a mixture of hardware provisioning and 


The resulting application is not only another \gls{iacacr} tool with \gls{toscaacr} as its chosen language. It is also the first tool that can provision bare-metal machines on demand.
\newline
The included \gls{toscaacr} orchestrator fulfills all requirements described in the specificiations of the standard and its Simple-Profile extension. Because the hardware extension requires low-level access to some host capabilites (like listening on hostport 67 and 68), the application needs to run with extended privileges. Additionally, the required software



what were the goals, what are the current capabilities
\newline
how stable is it, what can it do in real world settings
\newline
types, topology, tests
