\chapter{Design and Implementation} %  Design Principles and Architecture
%TODO add chapter introduction

% - 5-10 pages
% - goal: fellow student understands content and would be able to more or less reproduce work
% - legitimate chosen approach
% - develop own ideas, trace them to existing theories
% - analysis and development
% - why was the approach (algorithm/technique/...) chosen and how does it work
% - show how concepts from theory are applied
% - test setup and achieved results

% possible steps:
% - requirement study
% - analysis / design -> UML, interaction, behavioural model, basic algorithms/methods, detailed description of models and their interactions (class/sequence diagrams)
% - manual, how to use program/device
% - system development and implementation

% ---

% considerations:
% - ipv6 not 100 percent necessary, but would be good
% - easy to learn
% - easy to adapt (to counterpart-interface updates) -> plug-in system?
% - tosca-orchestrator:
%   - 4.3ff of \url{http://docs.oasis-open.org/tosca/TOSCA-Instance-Model/v1.0/csd01/TOSCA-Instance-Model-v1.0-csd01.html#_Toc500843787}
%     "orchestrators manage the state of nodes and transitions them from state to state. This notion of state is somewhat artificial in that the orchestrator assumes a stable state is reached after an operation executes [...] without error"
%     "an error results in an undefined state" (no automatic rollback defined in tosca)
%     "orchestration states are only valid during orchestration. [...] the orchestrator or the imperative workflow [...] must decide the current state of all nodes in the topology.
%     "[...] event stream can be maintained for the life of a deployment [...]"
%     "As nodes are transitioned through their states, a subset of attributes and relationships may be defined. [...] This requires that in general TOSCA implies semantics such that not all attributes would be available in a given state."
%     "Nodes are only visible when they have a state defined (i.e. the orchestrator is dealing with their lifecycle)"
%     "node attributes are only defined for the stable states"
%     "node relationships are always navigable when the source and target node exists"
%     "[...] state is never updated outside an orchestration. [...] no way to propagate state changes from the node to the orchestrator and nodes don't have a state attribute."
%       -> no state file!
%     "Nodes can update their attributes with no specific guarantees in terms of precision or accuracy"

\section{Requirements}
\begin{itemize}
  \item Understand basic TOSCA (CSAR files, file-structure, yaml-structure)
  \item Understand TOSCA-extensions (at least the one for hardware)
  \item wol capabilitiy
  \item ssh command execution
  \item feedback to user
\end{itemize}

\section{Architecture}
\begin{itemize}
  \item executable, library
  \item why and which modules/packages
  \item sub-commands (init, install, uninstall, ...)
\end{itemize}

\section{Features}
%TODO chapter introduction

\subsection{TOSCA package}

\subsection{CSAR package}

\subsection{Command-execution package}

\subsection{Docker control package}
creating images, running containers, stopping containers

\subsection{Live-OS image generation}
tests with different operating systems (size, speed, repeatable, webserver-capable)
\newline
live-debian details

\subsection{DHCP-, TFTP-, HTTP-server package}
docker container, ports, variable config vs fixed config

\subsection{Wake-on-lan package}
actual wol vs simulated wol for hypervisors

\subsection{SSH package}
key-generation, variable key path, key placement on servers
\newline
actual command execution and feedback returning

\subsection{TOSCA hardware extension}
types, topology, tests
